\section{Thermals}

A thermal is a finite patch of buoyant fluid from an isolated release.

The patch of buoyant fluid rises in the environment. It rises with speed $u$,
and its radius $r$ and volume $V$ increase due to entrainment. Its buoyancy
(with respect to a reference density $\rho_0$) is described by the reduced
gravity $g'$. The momentum of the thermal is $M \sim uV$.

Volume conservation and the entrainment hypothesis give
\begin{equation}
u \dod{V}{z} = 3(\alpha u) V^{2/3}
\end{equation}
where $\alpha u$ is the entrainment velocity.
The momentum balance states that 
\begin{equation}
u \dod{M}{z} = \frac{2}{3} V (g' - g'_e)
\end{equation}
and so
\begin{equation}
M \dod{M}{z} = \frac{2}{3} V^2 (g' - g'_e)
\end{equation}
where $g'_e$ is the reduced gravity of the environment, with respect to the
reference density $\rho_0$. The buoyancy balance gives
\begin{equation}
u \dod{B}{z} = 3 (\alpha u) V^{2/3} g'_e
\end{equation}
We have used $\od{}{t} = u\od{}{z}$. 

For a uniform environment with $g'_e = 0$, these equations can be solved, with
\begin{equation}
V = (V_0^{1/3} + \alpha z)^3 \text{ and } u = \frac{B^{1/2}}{\alpha^{3/2} z}
\end{equation}
and $r \sim \alpha z$. Experiments show that $\alpha = 0.4$.

When a plume starts, it is similar to a thermal. In fact, the top of the
starting plume behaves more like a thermal. This thermal is governed by the
above thermal equations, but with additional `plume forcing' terms:
\begin{align}
u\dod{V}{z} &= 3\alpha_s V^{2/3} + \beta q_p(z)  \\
\dod{(uV)}{z} &= \frac{2}{3} V g' \\
u\dod{(g'V)}{z} &= 3\alpha V^{2/3} g'_e + \beta q_p g'_p
\end{align}
where $q_p$ and $g'_p$ are the mass flux and reduced gravity of the plume, and
$\beta < 1$ is the fraction of the plume that enters the thermal. 

