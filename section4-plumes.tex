\section{Plumes}

\subsection{Self-similar plumes}
For plumes, the density of the fluid is not homogeneous, and so we need to consider fluxes of mass and momentum, rather than fluxes of volume and $\displaystyle\frac{\text{momentum}}{\text{density}}$. Therefore, we define
\begin{align}
 \pi Q	&= \frac{1}{T}\int_0^T \int_0^\infty \int_{-\pi}^\pi \hat\rho w r \dif \theta\dif r \dif t \\ 
 \pi M	&= \frac{1}{T}\int_0^T \int_0^\infty \int_{-\pi}^\pi \hat\rho w^2 r \dif \theta\dif r \dif t \\  
\end{align}
as the mass flux and momentum flux respectively. We also define
\begin{equation}
 \pi F		= \frac{1}{T}\int_0^T \int_0^\infty \int_{-\pi}^\pi (\rho_0 - \hat\rho) g w r \dif \theta\dif r \dif t \\ 
\end{equation}
as the buoyancy flux. Here $\hat\rho$ is the instantaneous, fluctuating density inside the plume, and $\rho_0$ is the density of the surrounding fluid. For top-hat profiles, these become
\begin{align}
 \pi Q	&= \pi\rho b^2 W \\
 \pi M	&= \pi\rho b^2 W^2 \\
 \pi F		&= \pi(\rho_0 - \rho) gb^2W
\end{align}
where $\rho$ is the density in the plume averaged across a cross-section. We also have
\begin{align}
 W		&= M/Q \\
 b		&= Q/(\rho M)^{1/2} \\
 g'		&= F/Q.
\end{align}

\subsection{Time-dependent plume equations}

Volume, mass and momentum balances on the plume give
\begin{align}
 \dpd{}{t} \left(\pi b^2\right) + \dpd{}{z} \left(\pi b^2 W\right) &= 2\pi b u_e \\
 \dpd{}{t} \left(\pi \rho b^2\right) + \dpd{}{z} \left(\pi \rho b^2 W\right) &= 2\pi \rho_0 b u_e \\
 \dpd{}{t} \left(\pi \rho b^2 W\right) + \dpd{}{z} \left(\pi \rho b^2 W^2\right) &= \pi b^2 (\rho_0 - \rho) g \\
\end{align}
which can also be written as 
\begin{align}
 \dpd{}{t} \left(\pi b^2\right) + \dpd{}{z} \left(\pi b^2 W\right) &= 2\pi b u_e \\
 \dpd{\rho}{t} + W\dpd{\rho}{z} &= \frac{2(\rho_0 - \rho) u_e}{b} \\
 \dpd{W}{t} + W\dpd{W}{z} &= \frac{g (\rho_0 - \rho)}{\rho} - 2\frac{\rho_0}{\rho} \frac{u_e W}{b}
\end{align}

Experiments show that the entrainment velocity for plumes is 
\begin{equation}
 u_e = \alpha \left(\frac{\rho}{\rho_0}\right)^{1/2} W
\end{equation}
where $\alpha$ is between $0.1$ and $0.16$ (but again depends on the top-hat
profile). For Boussinesq plumes, $\rho\sim\rho_0$ and so $u_e \sim \alpha W$. 

It can be shown that, in the Boussinesq case, the system is not hyperbolic, but
is parabolic, and admits similarity solutions. This is reasonable.

\section{Plumes in a homogeneous environment}

\subsection{Steady Boussinesq plumes}

Consider a steady injection of fluid of density $\rho+\Delta\rho$ into a
homogeneous environment of density $\rho$. The volume flux of the injection is
$Q$. 

The reduced gravity of the plume is $g' = g\Delta\rho/\rho_0$. 

\subsubsection{A simple plume model, by dimensional analysis}

The \textit{buoyancy flux} of the plume, $B$, is equal to $B_0 = g'_0 Q_0$ at
$z=0$. The dimensions of $B$ are $L^4 /T^3$.

Let $b = b(z)$ represent the radius of the plume at a height $z$. By dimensional
analysis, we must have $b = \lambda_1 z$.  Let $u = u(z)$ represent the mean
upwards speed of the plume. By dimensional analysis, we must have $u = \lambda_2
B^{1/3} z^{-1/3}$.  Therefore the volume flux $q = q(z)$ must be given by $q =
\lambda_3 B^{1/3} z^{5/3}$. Here, $\lambda_{1,2,3}$ are unknown, dimensionless
constants.

The buoyancy flux $B$ is conserved: $B = B_0$. This is because buoyancy is
caused by some conserved quantity, such as salt concentration or thermal energy.
Hence the reduced gravity will decrease according to $g'(z) = B_0 / q(z) =
\lambda_3^{-1} B_0^{2/3} z^{-5/3}$.

The $\lambda_i$ can now be found by experiments.

\subsubsection{Plume model with turbulence}

We now try to consider the effect of turbulence in the plume in more detail.
Consider a plume of radius $b(z)$ rising with upwards speed $u(z)$.  As the
plume rises, it entrains the surrounding fluid at the entrainment speed $u_e$.
The \textit{entrainment hypothesis} asserts that $u_e = \alpha u$ for some
$O(1)$ constant $\alpha$. This is a turbulence closure model.

In fact, the plume is not uniform across its area, and in fact does not have a
clear-cut radius. Rather, the velocity $u$ and concentration $\Delta c$ are given
by Gaussians, \textit{viz.}
\begin{align}
    u(r,z) &\sim u(0,z) \exp(-r^2/b_g^2) \\
    \Delta c(r,z) &\sim c(0,z) \exp(-r^2/\lambda b_g^2)
\end{align}
and the local density in the plume is given by $\rho = \rho_0(z) (1+\lambda \Delta c)$. The
ambient density $\rho_a(z) = \rho_0 (1+\lambda\Delta c_a)$.

The Boussinesq approximation now says that $\lambda\Delta c \ll 1$.

Mass conservation gives
\begin{equation}
    \dod{}{z} \int \dif A u \rho = 2\pi b_g K u(0,z) \rho_a(z).
\end{equation}
That is, the mass flux in the plume increases according to the entrainment of
ambient fluid. Here $b_g$ is the `Gaussian radius', and $K u(0,z)$ is an
entrainment velocity. (We will relate the constants $\alpha$ and $K$ later.)

Buoyancy conservation (which as above comes about because of the conservation of
some quantity which causes buoyancy, such as salt or thermal energy) gives
\begin{equation}
    \dod{}{z} \int \dif A u\rho\Delta c = 2\pi b_g K u(0,z) \rho_a (z) \Delta c_a
\end{equation}
In an unstratified environment, $\Delta c_a = 0$.

The equation of motion asserts that
\begin{equation}
    \rho (\bs{u}\cdot\grad)\bs{u} = -\grad p + \rho\bs{g}.
\end{equation}
Integrating the $z$-component over the volume between $z$ and $z + \Delta z$ gives
\begin{equation}
    \int\dif V \rho (\bs{u}\cdot\grad)\bs{u}\cdot\bs{e}_z = \int (\rho_a - \rho) g \dif V.
\end{equation}
That is, the difference between the plume density and the ambient density is
responsible for changes in the volume flux. We now use the fundamental theorem
of calculus to obtain
\begin{equation}
    \dod{}{z} \int \rho u^2 \dif A = \int (\rho_a - \rho) g \dif A
\end{equation}
where these are surface integrals over a horizontal cross-section of the plume.

The Boussinesq approximation $\lambda\Delta c \ll 1$ gives $\rho,\rho_a \approx
\rho_0$, and so we have
\begin{equation}
    \dod{}{z} \int u \dif A = 2\pi b_g K u(0,z),
\end{equation}
a volume conservation equation, with entrainment. From conservation of $c$-flux,
we have
\begin{equation}
    \dod{}{z} \int u \Delta c \dif A = 2\pi b_g K u(0,z) \Delta c_a.
\end{equation}
However, in the conservation of momentum equation, we take $\rho_a - \rho$, and
the leading-order terms cancel out. Rather,
\begin{equation}
    \dod{}{z} \int u^2 \dif A = \int g \lambda (\Delta c_a - \Delta c) \dif A = \int g' \dif A
\end{equation}
where $g' = g\lambda(\Delta c_a - \Delta c)$ is the reduced gravity.

We now pretend that the plume's velocity and buoyancy functions have top-hat
profiles of radius $b$, rather than Gaussian profiles with characteristic radius
$b_g$. This introduces constant conversion factors which will relate $b$ and
$b_g$. We assume that the velocity, buoyancy and concentration are uniform
across each cross-section, and zero outside the top hat. We now put bars above
the names of variables, to stress that we are referring to \textit{averaged}
quantities: For example, we will call the velocity $\bar{u}$, to represent the
mean velocity inside the plume.

The integrals above can therefore be simplified, and the integral equatinos can
be written as differential equations instead:
\begin{align}
    \dod{}{z}(\bar{u} b^2) &= 2\alpha b\bar{u} \\
    \dod{}{z}(\bar{u}^2 b^2) &= \bar{g'} b^2 \\
    \dod{}{z}(\bar{u} b^2 \bar{\Delta c}) &= 2\alpha b\bar{u} \Delta c_a
\end{align}
Multiply the first of these by $-\Delta c_a$ and add the third equation:
\begin{equation}
    \dod{}{z} (\bar{u} b^2 (\bar{\Delta c} - \Delta c_a)) = -\bar{u} b^2 \dod{\Delta c_a}{z} 
\end{equation}
Multiplying by $g \lambda$ lets us write this in terms of the reduced gravity:
\begin{equation}
    \dod{}{z}(\bar{u} b^2 \bar{g'}) = -N^2 \bar{u} b^2
\end{equation}
where $N^2 = -(g/\rho_0) \pd{\rho_a}{z}$ is the square of the
Brunt--V\"ais\"al\"a frequency.

For an unstratified environment with $N^2 = 0$, we have
\begin{align}
    \dod{}{z}(\bar{u} b^2) &= 2\alpha b\bar{u} \\
    \dod{}{z}(\bar{u}^2 b^2) &= \bar{g'} b^2 \\
    \dod{}{z}(\bar{u} b^2 \bar{\Delta c}) &= 0
\end{align}
Define $q = \bar{u}b^2$ as the (specific) volume flux, $m = \bar{u}^2 b^2$ as
the (specific) momentum flux, and $B = \bar{u}b^2 \bar{g'}$ as the (specific)
buoyancy flux.\footnote{The word `specific' refers to the fact that we are
ignoring any factors of $\pi$, $\rho_0$, etc.} Then we have
\begin{align}
    \dod{q}{z} &= 2\alpha m^{1/2} \\
    m\dod{m}{z} &= B q \\
    \dod{B}{z} &= 0
\end{align}
and so the buoyancy flux $B = B_0$ is constant in an unstratified environment.
Multiplying the second equation by $2\alpha m^{1/2}$ gives us
\begin{equation}
    2\alpha m^{3/2} \dod{m}{z} = B_0 q\dod{q}{z}
\end{equation}
which integrates to
\begin{equation}
    m^{5/2} = \left(\frac{5B_0}{8\alpha}\right) q^2 + \left( m_0^{5/2}
    - \left(\frac{5B_0}{8\alpha}\right) q_0^2 \right) 
    = \left(\frac{5B_0}{8\alpha}\right) q^2 + J
\end{equation}
for some value of $J$.

\paragraph{The case $J=0$} A plume for which $J = 0$ is called a \textit{pure
plume}.In this case we can write exact solutions for $b$ and $q$, and therefore
for $\bar{u}$ and for $\bar{g'}$. These solutions scale with $z$ exactly as the
dimensional arguments predicted:
\begin{align}
    \bar{u} &= \left(\frac{5}{6\alpha}\right) \left( \frac{9\alpha B_0}{10}\right)^{1/3} z^{-1/3} \\
    \bar{g'} &= \left(\frac{5B_0}{6\alpha}\right) \left( \frac{9\alpha
        B_0}{10}\right)^{-1/3} z^{-5/3} \\
        q &= \left(\frac{6\alpha}{5}\right)^{5/3}
        \left(\frac{5B_0}{8\alpha}\right)^{1/3} z^{5/3}
\end{align}
We can now test our model by conducting an experiment and finding whether the
volume flux $q$ does scale with $z$ as predicted. We will need to conduct an
experiment to find the empirical constant $\alpha$.

\paragraph{The case $J>0$} With $J>0$, the plume initially has more momentum
than a pure plume. Near the source, the plume will behave more like a jet,
entraining surrounding fluid more quickly than a pure plume. Such a plume is
called a \textit{forced plume}. As the surrounding fluid is entrained, the
starting momentum becomes less important and the forced plume adjusts to become
like a pure plume.

We define the \textit{jet length} $L_j$ as the distance over which the momentum is
important compared to the buoyancy. By dimensional arguments, we have $L_j \sim
M_0^{1/4} B_0^{-1/2}$.

\paragraph{The case $J<0$} In this case the volume flux is too great compared to
the buoyancy flux. Such a plume is called a \textit{distributed plume}. The flow
will adjust to become like a pure plume by accelerating, over a length scale
$L_a \sim Q_0^{3/5} B_0^{-1/3}$ called the \textit{acceleration length}. A
distributed plume does not entrain surrounding fluid as quickly as a pure plume.

\paragraph{Summary} The case $J=0$, known as \textit{pure plume balance}, is a
balance between volume, momentum and buoyancy fluxes. 

\subsection{Time-dependent plumes}
\subsection{Steady non-Boussinesq plumes}

\section{Plumes in a stratified environment}

Now the ambient density is a function of height, and $N^2 \neq 0$. The equations
become 
\begin{align}
    \dod{q}{z} &= 2\alpha m^{1/2} \\
    m\dod{m}{z} &= B q \\
    \dod{B}{z} &= -N^2 q
\end{align}
Suppose the stratification is uniform, so that $N^2$ is constant.

\subsection{Rise height}

What determines the rise height $H$? Since $H = f(N, B_0)$ and using dimensional
analysis, we have 
\begin{equation}
    H = 5 B^{1/4} N^{-3/4} 
\end{equation}
where the number $5$ must be obtained from experiment (either directly or after
working out $\alpha$ from experiment).

The differential equations for $q$, $m$ and $B$ have no analytic solution; we
must use numerical methods to solve them. It is found that $b$ initially grows
linearly with $z$ but then tends towards a constant; $\bar{u}$ decreases and
eventually goes to 0, and $\bar{g'}$ decreases and eventually becomes negative.

The height at which $B = 0$ is called the \textit{neutral height}. Above the
neutral height, $B < 0$ and so the momentum drops, eventually to 0. The height
at which $\bar{u} = 0$ is called the \textit{top height}, and is higher than the
neutral height.

Thus, a parcel of fluid entering the plume will rise, above the neutral height,
reach the top height, and then fall back down to the neutral height. Above the
neutral height, the shape of the plume looks like a fountain. Strictly speaking,
our model does not work here, because the entrainment above the neutral height
is different from the entrainment below. That said, the numerical results agree
with experimental results very well.

\subsection{In air and water}

\paragraph{Thermodynamics} For air, temperature and density $\rho = \rho(T)$ are
related by $\rho = P/RT$ where $R$ is the ideal gas constant. Thus
$\rho(T+\Delta T) = (P/RT) (1-\Delta T/T+\dots)$ and so $g' \sim \Delta T/T$. 

\paragraph{Hydrothermal plumes} In the ocean, $\rho$ is governed by both
salinity and temperature: $\rho = \rho(S,T)$. At leading order, $\rho(S+\Delta
S, T+\Delta T) = \rho_0 (1+\alpha_S \Delta S + \alpha_T \Delta T + \dots)$.

Since both heat and salt must be conserved, we have four equations in our model:
\begin{align}
    \dod{q}{z} &= 2\alpha m^{1/2} \\
    m\dod{m}{z} &= (B_S + B_T) q \\
    \dod{B_S}{z} &= -N_S^2 q \\
    \dod{B_T}{z} &= -N_T^2 q
\end{align}
where $B_S$ is the buoyancy due to salinity and $B_T$ is the buoyancy due to
heat, and $N_S$ and $N_T$ are the two different buoyancy frequencies.

In some parts of the ocean, the ocean may be stably stratified with respect to
salinity but unstably stratified with respect to temperature, or vice-versa.
Overall, the ocean may still be stably stratified. But the cause of buoyancy for
the plume may be different at different depths.

\subsection{Series solution in Boussinesq fluid}

\subsection{Plumes in a confined space and ventilation}



