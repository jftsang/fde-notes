\newpage
\section{Jets, plumes and thermals}

A \textit{jet} is a continuous directed release of fluid with a specified momentum from an isolated source (such as blowing from lips). A \textit{plume} is a continuous release of buoyant fluid from an isolated source, with, ideally, no mass or momentum flux (e.g. a candle). The direction of a plume is given by $\bs{g}$. A \textit{thermal} is an isolated release of a finite patch of buoyancy.

Note that `continuous', `stable' and `steady' are distinct terms, and no one of them implies either of the other. 

We will assume that jets and plumes are axisymmetric and self-similar, so that any averaged quantity $\bar\phi$ can be written as
\begin{equation}
 \bar\phi = \Phi(z) f(r/b)
\end{equation}
where $b$ is the `radius' of the plume. The averaging may be an ensemble average or a temporal average. Specific forms for the function $f$ may be taken. The simplest form for $f$ is the \textit{top-hat} profile
\begin{equation}
 f(x) = \begin{cases} 
  0 & \text{if } x>1\\
  1 & \text{if } x<1.
 \end{cases}
\end{equation} 
Other forms, such as the Gaussian $f(x) = \exp(-x^2)$, may also be used.

\subsection{Jets}

We define 
\begin{align}
 \pi Q	&= \frac{1}{T}\int_0^T \int_0^\infty \int_{-\pi}^\pi w r \dif \theta\dif r \dif t \\ 
 \pi M	&= \frac{1}{T}\int_0^T \int_0^\infty \int_{-\pi}^\pi w^2 r \dif \theta\dif r \dif t \\  
\end{align}
as, respectively, the volume flux and momentum flux divided by density. The integral over $t$ enacts a time-averaging. The $\pi$'s on the LHS are for convenience. For a top hat profile, these become
\begin{align}
 \pi Q	&= \pi b^2 W \\
 \pi M	&= \pi b^2 W^2 
\end{align}
where we write $W$ instead of $w$ to emphasise that this is an averaged velocity, reserving $w$ for the fluctuating local velocity. We have
\begin{align}
 W	&= M/Q \\
 b	&= Q/M^{1/2}
\end{align}

Consider a steady jet travelling in the $z$ direction. As the fluid in the jet travels, it entrains the surrounding fluid with entrainment velocity $u_e$. Experiments find that
\begin{equation}
 u_e = \alpha W 
\end{equation} 
where the entrainment coefficient $\alpha$ is between 0.065 and 0.080 for a top hat profile. Thus, the volume flux satisfies
\begin{align}
 \pi \dod{Q}{z}	&= 2\pi b u_e \\
 			&= 2\pi b \alpha W \\
			&= 2\alpha\pi M^{1/2}.
\end{align}
Meanwhile, conservation of momentum gives
\begin{equation}
 \pi \dod{M}{z} = 0.
\end{equation}
Hence $M = M_0$ is constant along the plume. This gives
\begin{align}
 Q 	&= 2\alpha M_0^{1/2} (z+z_\nu) \\
 b	&= \alpha (z+z_\nu) \\
 W	&= \frac{M_0^{1/2}} {2\alpha (z+z_\nu)}.
\end{align}
So, the jet's thickness increases linearly with $z$, and the fluid velocity decreases with $z$. The point $z = -z_\nu$ is the `virtual origin' of the jet: it looks as though there is a point source of momentum at this point, even though boundary conditions are being specified at $z=0$. 

\subsection{General plume equations}

\subsubsection{Self-similar plumes}
For plumes, the density of the fluid is not homogeneous, and so we need to consider fluxes of mass and momentum, rather than fluxes of volume and $\displaystyle\frac{\text{momentum}}{\text{density}}$. Therefore, we define
\begin{align}
 \pi Q	&= \frac{1}{T}\int_0^T \int_0^\infty \int_{-\pi}^\pi \hat\rho w r \dif \theta\dif r \dif t \\ 
 \pi M	&= \frac{1}{T}\int_0^T \int_0^\infty \int_{-\pi}^\pi \hat\rho w^2 r \dif \theta\dif r \dif t \\  
\end{align}
as the mass flux and momentum flux respectively. We also define
\begin{equation}
 \pi F		= \frac{1}{T}\int_0^T \int_0^\infty \int_{-\pi}^\pi (\rho_0 - \hat\rho) g w r \dif \theta\dif r \dif t \\ 
\end{equation}
as the buoyancy flux. Here $\hat\rho$ is the instantaneous, fluctuating density inside the plume, and $\rho_0$ is the density of the surrounding fluid. For top-hat profiles, these become
\begin{align}
 \pi Q	&= \pi\rho b^2 W \\
 \pi M	&= \pi\rho b^2 W^2 \\
 \pi F		&= \pi(\rho_0 - \rho) gb^2W
\end{align}
where $\rho$ is the density in the plume averaged across a cross-section. We also have
\begin{align}
 W		&= M/Q \\
 b		&= Q/(\rho M)^{1/2} \\
 g'		&= F/Q.
\end{align}

\subsubsection{Time-dependent plume equations}

Volume, mass and momentum balances on the plume give
\begin{align}
 \dpd{}{t} \left(\pi b^2\right) + \dpd{}{z} \left(\pi b^2 W\right) &= 2\pi b u_e \\
 \dpd{}{t} \left(\pi \rho b^2\right) + \dpd{}{z} \left(\pi \rho b^2 W\right) &= 2\pi \rho_0 b u_e \\
 \dpd{}{t} \left(\pi \rho b^2 W\right) + \dpd{}{z} \left(\pi \rho b^2 W^2\right) &= \pi b^2 (\rho_0 - \rho) g \\
\end{align}
which can also be written as 
\begin{align}
 \dpd{}{t} \left(\pi b^2\right) + \dpd{}{z} \left(\pi b^2 W\right) &= 2\pi b u_e \\
 \dpd{\rho}{t} + W\dpd{\rho}{z} &= \frac{2(\rho_0 - \rho) u_e}{b} \\
 \dpd{W}{t} + W\dpd{W}{z} &= \frac{g (\rho_0 - \rho)}{\rho} - 2\frac{\rho_0}{\rho} \frac{u_e W}{b}
\end{align}

Experiments show that the entrainment velocity for plumes is 
\begin{equation}
 u_e = \alpha \left(\frac{\rho}{\rho_0}\right)^{1/2} W
\end{equation}
where $\alpha$ is between $0.1$ and $0.16$ (but again depends on the top-hat profile). For Boussinesq plumes, $\rho\sim\rho_0$ and so $u_e \sim \alpha W$. 

It can be shown that, in the Boussinesq case, the system is not hyperbolic, but is parabolic, and admits similarity solutions, as expected. 

\subsection{Plumes in a homogeneous environment}
\subsubsection{Steady Boussinesq plumes}

It is found that $F$ is constant. 

\subsubsection{Time-dependent plumes}
\subsubsection{Steady non-Boussinesq plumes}

\subsection{Plumes in a stratified environment}

Now the ambient density $\rho_0$ is a function of height. This must be accounted for in the equation for $F$. It is found that $F$ is not constant, and a power-law solution does not exist.

\subsubsection{Rise height}
\subsubsection{Series solution in Boussinesq fluid}

\subsection{*Thermal in a homogeneous environment}
\subsection{*Thermals in a stratified environment}