\newpage
\chapter{Turbulence primer} \label{section2}

Most environmental and geophysical flows occur at high Reynolds numbers. At high
Reynolds numbers, steady laminar solutions to the Navier-Stokes equations are
extremely unstable, and the flows are turbulent. In transitioning from a steady
laminar solution to a fully turbulent solution, a system undergoes a number of
bifurcations as the Reynolds number increases beyond critical values. These
bifurcations are analogous to Hopf bifurcations, or to the period-doubling
bifurcations exhibited by the logistic map.

\section{Basics of turbulence}

Throughout this and the next section, we shall assume that the density $\rho$ is
constant and uniform.

\subsection{What is turbulence?}

Although there is no simple definition for turbulence, turbulent flows can be
characterised by a few properties:
\begin{itemize}
    \item Turbulent flows are dominated by inertia, since $\mathrm{Re}\gg1$.
        Viscosity plays a small (but crucial) role.
    \item Turbulent flows are vortical, not irrotational. However, vortical
        flows need not be turbulent.
    \item Turbulent flows involve a range of length scales. Vortices exist at
        both large and small length scales: ``Big whorls have little whorls,
        which feed on their velocity, // And little whorls have lesser whorls,
        and so on to viscosity.'' ---Lewis Richardson.
\end{itemize}
Turbulent flows are unpredictable and simulating them numerically may be
difficult as they tend to be numerically unstable. However, it may be possible
to predict certain statistics of the flow, such as the behaviour of the mean
flow or mean pressure (noting that fluctuations about these means may be great).
Some features of turbulence can be shown to be universal, and independent of
geometry.

Lewis Richardson's couplet describes the two processes at work in a turbulent
flow. \textit{Vortex stretching} is responsible for creating small vortices out
of large ones, while \textit{energy dissipation} causes small vortices to decay.

\subsection{Vortex stretching}

Vortex stretching is the main process by which energy is transferred from large
scales to smaller scales.

Taking the curl of the Navier-Stokes equation gives the \textit{vorticity equation}:
\begin{equation}
    \pd{\omega}{t} + \bs{u}\cdot\grad\bs{\omega} 
     = \bs{\omega}\cdot\grad\bs{u} + \nu\grad^2\bs{\omega}
 \label{vorticity-eqn}
\end{equation}
The second term on the RHS of (\ref{vorticity-eqn}) means that vorticity is
diffused with diffusion constant $\nu$. This is a minor effect in flows with
$\mathrm{Re}\gg1$. The first term means that when a parcel of fluid is stretched
into a column, its vorticity increases and becomes aligned with the column. Thus
stretching a vortex takes energy to a smaller scale. However, if the vortex
column is compressed, not nearly as much energy is taken back to the large
scale, since the compression in general de-aligns the vorticity vector and the
column.

\subsection{Energy dissipation: Kolmogorov's theory (1941)}

We know that energy is dissipated in a fluid at a rate
\begin{equation}
	\Delta = \int_V\mu|\grad\bs{u}|^2dV
\end{equation}
(or expressed in terms of the strain rate tensor, using $|\grad\bs{u}|^2=2e_{ij}e_{ij}$). In a volume with characteristic length $L$, 
\begin{equation}
	\Delta \sim L^3 |\grad\bs{u}|^2.
\end{equation}
For a laminar flow with characteristic velocity $U$ in a box with sides $L$, $|\grad\bs{u}|\sim U/L$ and so
\begin{equation}
\Delta \sim \mu\frac{U^2}{L^2}L^3 = \mu U^2 L.
\end{equation}  
However, for a turbulent flow, the energy dissipation is much faster than this, because the velocity gradients are over much smaller distances, and so $|\grad\bs{u}|\gg U/L$. 

Although the large-scale structure of the flow might have characteristic
lengthscale $L$, the small-scale structure of the flow should be described by
another lengthscale $\eta$, called the \textit{Kolmogorov length scale} or the
\textit{Kolmogorov microscale}. This is the lengthscale at which the effects of
vortex stretching and dissipation balance each other.

To summarise: The flow is externally driven (continually or initially) at the
lengthscale $L$, energy cascades to smaller lengthscales by vortex stretching;
viscosity acts more and more strongly at these smaller lengthscales. Eventually,
energy cascades down to the lengthscale $\eta$ and proceeds no more.

Let $\bs{U}(\bs{k})$ be the Fourier transform of the velocity $\bs{u}(\bs{x})$, using the normalisation
\begin{equation}
    \bs{U}(\bs{k}) = \frac{1}{\sqrt{2\pi}} \int_V \bs{u}(\bs{x}) e^{-i\bs{k}\cdot\bs{x}} d\bs{x}.
\end{equation}
Let 
\begin{equation}
    E(\bs{k}) = \frac{1}{2}\rho\bs{U}(\bs{k})\cdot\bs{U}(\bs{k})*
\end{equation}
be the energy density spectrum. (In defining these Fourier transforms, we are assuming that $\rho$ is constant and uniform.) The total kinetic energy is
\begin{equation}
    KE = \int E(\bs{k}) d\bs{k}
\end{equation}
with the integral taken over wavenumber space. For isotropic turbulence, $\bs{U}$ and $E$ depend only on the magnitude $k=|\bs{k}|$. 

Since density is constant, we will find it convenient to rewrite the above expressions without $\rho$, so that the dimension $M$ may be set to $1$. We therefore work with the energy dissipation rate 
\begin{equation}
    \epsilon = \int_V\nu|\grad\bs{u}|^2dV
\end{equation}
which has dimensions $L^2 T^{-3}$, and with the energy spectrum
\begin{equation}
    E(k) = \frac{1}{2}\bs{U}(k)\cdot\bs{U}(k)*
\end{equation}
which has dimensions $L^3 T^{-2}$. We write 
\begin{equation}
    KE = \int E(k) dk
\end{equation}
for the kinetic energy per unit mass, with dimensions $L^2 T^{-2}$.

Kolmogorov made two hypotheses about isotropic turbulent flows. These hypotheses allow us to make progress by dimensional analysis, but they do not follow from first principles, and their predictions must be verified experimentally.

\paragraph{First hypothesis} The rate of dissipation of energy (per unit density), $\epsilon$ (dimensions $L^2T^{-3}$), depends only on viscosity $\nu$ (dimensions $L^2T^{-1}$) and on the microscale $\eta$. In particular, $\epsilon$ and $\eta$ are not to depend on the large lengthscale $L$.

By dimensional analysis, this hypothesis allows us to construct $\eta$ (up to some constant):
\begin{equation}
    \eta = (\nu^3 / \epsilon)^{1/4}
\end{equation}

Note that in a system in dynamic equilibrium, the rate of dissipation of energy is equal to the rate of energy input.
\footnote{A flow in dynamic equilibrium is unsteady, but it can be characterised by statistics which are constant in time. For example, the mean flow may be zero and the rms velocity may be constant. A turbulent flow may not be in static equilibrium, but it may be in dynamic equilibrium.}

\paragraph{Second hypothesis} The energy density $E(k)$ (dimensions $L^3T^{-2}$) depends only on the rate of dissipation $\epsilon$ and on $k$ (dimensions $L^{-1}$).

By dimensional analysis,
\begin{equation}
    E(k) = C \epsilon^{2/3} k^{-5/3}
    \label{kolmogorov-energy-spectrum}
\end{equation}
for some constant $C$.

This second hypothesis is not completely valid. If the flow is in a bounded domain with characteristic lengthscale $L$, or if the flow is driven at the lengthscale $L$, then the hypothesis is not valid at wavenumbers $k\sim1/L$ or below. And the hypothesis is also not valid at small scales with $k\sim1/\eta$, where the details of dissipation start to be important. However, for the range $1/L\ll k\ll 1/\eta$, known as the \textit{inertial range}, the hypothesis is justified and is supported by experimental evidence: (\ref{kolmogorov-energy-spectrum}) is reproduced by experimental data. 

\subsection{Two-dimensional turbulence}

A turbulent flow in two dimensions is different from one in three dimensions, because in two dimensions the vortex stretching term in (\ref{vorticity-eqn}) is always zero. Vorticity is conserved, and there is no cascade of energy to smaller lengthscales. Dotting (\ref{vorticity-eqn}) with
$\bs{\omega}$ gives us the \textit{enstrophy equation}:
\begin{equation}
    \DDt{}\left(\frac{1}{2}|\bs{\omega}|^2\right) 
    = \nu\left(\grad^2 \left( \frac{1}{2}|\bs{\omega}|^2 \right) - |\grad\bs{\omega}|^2\right).
 \label{enstrophy-eqn}
\end{equation}
The scalar quantity $\frac{1}{2}|\bs{\omega}|^2$ is called the \textit{enstrophy density}. 

We obtain a very similar equation for the kinetic energy density by dotting the
Navier-Stokes equation with $\bs{u}$:
\begin{equation}
    \DDt{}\left(\frac{1}{2}|\bs{u}|^2\right) 
    = \nu\left(\grad^2 \left( \frac{1}{2}|\bs{u}|^2 \right) - |\grad\bs{u}|^2\right).
 \label{ke-eqn}
\end{equation}
Equations \ref{enstrophy-eqn} and \ref{ke-eqn} tell us that in a 2D flow, enstrophy and kinetic energy are advected, redistributed by diffusion (first term on RHS), and dissipated by viscosity (second term on RHS).

Since total enstrophy (enstrophy density integated across the volume) must
decrease, there must be a net movement of energy to \textit{larger} lengthscales
(i.e.  smaller $k$). Total enstrophy is proportional to $\int k^2E(k) dk$, so if
$E(k)$ increases at high $k$, this increase must be balanced by a greater
increase in $E(k)$ at low $k$, so that $\int k^2E(k) dk$ never increases.

Hence there is an \textit{anticascade} of energy to larger scales, and a
\textit{cascade} of enstrophy to smaller scales.

\section{Simplistic approaches to modelling turbulence}

Instead of starting from first principles and trying to solve the Navier-Stokes equations for a turbulent system directly, it is often easier, and more fruitful, to model the effects of turbulence instead.

\subsection{Molecular diffusion}

A direct approach is to model all scales down to the microscale
$\eta=(\nu^3/\epsilon)^{1/4}$. The energy spectrum
(\ref{kolmogorov-energy-spectrum}) predicts that the kinetic energy 
per unit volume will be 
\begin{equation}
    \text{KE density} = \int_{1/L}^{1/\eta} E(k) dk \propto (\epsilon L)^{2/3}
\end{equation}
giving a characteristic velocity
\begin{equation}
    u \propto \sqrt{\text{KE density}} \propto (\epsilon L)^{1/3}
\end{equation}
and therefore a Reynolds number
\begin{equation}
    Re = \frac{uL}{\nu} \propto \frac{\epsilon^{1/3} L^{4/3}}{\nu}.
\end{equation}
Note that we calculate this Reynolds number using a characteristic velocity
calculated only from the turbulent properties, not taking into account any
large-scale velocities (such as one driving the flow). 

Since $\eta = (\nu^3/\epsilon)^{1/4}$, we have
\begin{equation}
    Re = \left(\frac{L}{\eta}\right)^{4/3} \gg 1.
\end{equation}

For moderately large $Re$, direct numerical simulation (DNS) may be possible,
but usually $Re$ is so large that DNS is impractical or impossible. 

The problem would be even worse if the fluid were not of uniform and constant
density. The mass diffusivity $\kappa$ and the momentum diffusivity $\nu$ are
compared by the \textit{Schmidt number} $Sc = \nu/\kappa$. For salt water, $Sc
\approx 1000$. In order to simulate the diffusion of mass, we would have to look
at an even smaller scale, the \textit{Batchelor scale}
$\lambda_B=\eta/Sc^{1/2}$. 

\subsection{The closure problem}

A turbulent flow $\bs{u}$, and the associated pressure field $p$, can be decomposed into its mean part $\overline{\bs{u}}$, and a fluctuating part $\bs{u}'$ that fluctuates about $\bs{0}$: $\overline{\bs{u}'} = \bs{0}$. We can model a turbulent flow by ignoring the details of these fluctuations, and looking only at how these fluctuations affect the mean part.\footnote{The mean can be taken as a time average, a local spatial average, or as an ensemble average, but this detail is unimportant for what follows.}

Beginning with the Navier-Stokes equations:
\begin{equation}
    \rho\DDt{\bs{u}} = -\grad p + \mu\grad^2\bs{u} + \bs{f}
    \label{ns-eqn}
\end{equation}
and averaging, we obtain
\begin{equation}
    \rho\left(\frac{\partial\overline{\bs{u}}}{\partial t} + \overline{\bs{u}}\cdot\grad\overline{\bs{u}}\right)
    = -\grad\overline{p} + \mu\grad^2\overline{\bs{u}} + \bs{f} - \rho\overline{\bs{u}'\cdot\grad\bs{u}'}
    \label{ave-ns-eqn}
\end{equation}
(assuming that the body force $\bs{f}$ is a constant). The fluctuations are still present in the quadratic term $-\rho\overline{\bs{u}'\cdot\grad\bs{u}'}$ on the RHS, because (\ref{ns-eqn}) is nonlinear.\footnote{If we had a non-constant $\rho$, then we would have further such quadratic terms, such as the turbulent buoyancy flux $\overline{\bs{u}'\cdot\grad\rho}$.} 
We cannot remove such terms by more averaging, as that simply introduces further fluctuating terms. This is the \textit{closure problem}: we cannot close the system as we do not know how to deal with these terms. Many approximate closures have been proposed, but none are universally applicable.

Note that 
\begin{equation}
    -\rho\overline{\bs{u}'\cdot\grad\bs{u}'} = -\rho\grad\cdot(\overline{\bs{u'}\bs{u'}})
\end{equation}
and so the fluctuations affect the mean flow as though they were an additional
stress force, with stress tensor 
\begin{equation}
    \bs{\sigma}^{(R)} = -\rho \overline{\bs{u'}\bs{u'}}.
\end{equation}
This `stress' is called the \textit{Reynolds stress}. Different closure models propose different formulae for the Reynolds stresss, usually in terms of some properties of the mean flow. 

\subsection{The $k$-$\epsilon$ model}

The most commonly used closure model is the $k$-$\epsilon$ model, where
$k=\frac{1}{2}\bs{u}'\cdot\bs{u}'$ refers to the \textit{turbulent kinetic energy} (per
unit density), and $\epsilon$ to the energy dissipation. The model states that
\begin{equation}
    \rho\DDt{k}
    = \grad\cdot\left(\frac{\mu_T}{\sigma_k}\grad k\right) 
    + 2 \mu_T \bs{S}:\bs{S} 
    - \rho\epsilon
    \label{ke-model-1}
\end{equation}
where 
\begin{equation}
    \mu_T = C_\mu\rho\frac{k^2}{\epsilon}
\end{equation}
is the \textit{turbulent viscosity}, 
\begin{equation}
    \bs{S} = \frac{1}{2}\left[\grad\overline{\bs{u}} + (\grad\overline{\bs{u}})^T\right]
\end{equation}
is the strain rate tensor corresponding to the mean flow, and $C_\mu \approx
0.09$ and $\sigma_k\approx1.00$ are constants that are determined empirically.
Equations \ref{ave-ns-eqn} and \ref{ke-model-1} together specify the problem for
$\overline{\bs{u}}$ completely.

Essentially, the model describes how turbulent kinetic energy is governed. We
can interpret (\ref{ke-model-1}) as follows:
\begin{itemize}
    \item The first term on the RHS represents the diffusion of turbulent
        kinetic energy, at some (non-constant) rate $\mu_T/\sigma_k$.
    \item The second term represents the generation of turbulent kinetic energy,
        by means of vortex stretching.
    \item The third term represents the dissipation of turbulent kinetic energy.
\end{itemize}

Equation \ref{ke-model-1} can be re-written as 
\begin{equation}
    \rho\DDt{\epsilon} =
    \grad\cdot\left(\frac{\mu_T}{\sigma_\epsilon}\grad\epsilon\right)
    + 2 C_{1\epsilon} \frac{\epsilon}{k}\mu_T\bs{S}:\bs{S}
    - C_{2\epsilon}\rho\frac{\epsilon^2}{k}
\end{equation}
where $\sigma_\epsilon\approx1.30$, $C_{1\epsilon}\approx1.44$ and
$C_{2\epsilon}\approx1.92$ are constants that are determined empiracally.

Experiments and simulations find that these constants are \textit{universal}.

\subsection{Turbulent diffusion models}

As discussed, the vortex stretching process is irreversible and takes energy to
smaller lengthscales. There is little back-scatter: smaller lengthscales do not
strongly affect the larger scales, except as a sink of energy. This is why we
can consider the mean flow as in (\ref{ave-ns-eqn}) and ignore the behaviour of
the fluctuations, except as a sink of energy.

This suggests replacing the Reynolds stress with an additional viscosity term
that mimics the diffusion of momentum and dissipation of energy without actually
describing how this happens. That is, we take
\begin{equation}
    -\overline{\bs{u}' \bs{u}'} = \nu_T \bs{S}
\end{equation}
where $\nu_T$ is called the \textit{(turbulent) eddy viscosity}. We also write
$\mu_T = \rho\nu_T$. The force due to Reynolds stresses is therefore
\begin{align}
    \divg\bs{\sigma}^{(R)} &= \divg\left(-\rho\overline{\bs{u}'\bs{u}'}\right)  \\
     &= \rho\grad\cdot(\nu_T\bs{S}) \\
     &= \rho (\grad\nu_T \cdot \bs{S} + \nu_T \divg\bs{S}) \\
     &= \rho \left(\grad\nu_T \cdot \bs{S} + \frac{1}{2} \nu_T \grad^2\overline{\bs{u}}\right).
\end{align}
Putting this into the averaged Navier-Stokes equations gives us
\begin{equation}
    \left(\frac{\partial\overline{\bs{u}}}{\partial t} +
    \overline{\bs{u}}\cdot\grad\overline{\bs{u}}\right)
    = -\frac{1}{\rho}\grad\overline{p} 
    + \left(\nu+\frac{1}{2}\nu_T\right)\grad^2\overline{\bs{u}} 
    + \frac{1}{\rho}\bs{f} 
    + \grad\nu_T \cdot \bs{S}.
    \label{ave-ns-eqn-td-model}
\end{equation}
Closure models that do this are called \textit{turbulent diffusion models}.
Different models propose different expressions for $\nu_T$: this is usually not 
constant, but may depend on the position or on $\overline{\bs{u}}$.

\subsection{Prandtl's mixing length model}

Prandtl's mixing length model is a turbulent diffusion model, in which
\begin{equation}
    \nu_T = k u' l
\end{equation}
where $l$ is some \textit{mixing length}, $u'=\sqrt{\overline{|\bs{u}'|^2}}$
is the \textit{turbulence intensity}, and the \textit{Prandtl ratio}
$k\approx0.4$ is another empirically-determined constant. The mixing length
represents the lengthscale of turbulent eddies; it may vary with position, and
depends on the geometry of the problem. 

\paragraph{The law of the wall} For example, in the domain $z>0$ bounded by a
wall at $z=0$, the lengthscale $l$ of turbulent eddies is assumed to scale with
$z$. The turbulence intensity is assumed to be some constant $q$. Hence $\nu_T =
kqz$.  The horizontal shear stress is also assumed to be constant. 

Assuming that the mean flow is $\overline{\bs{u}} = U(z) \bs{e}_x$, the
$x$-component of Equation \ref{ave-ns-eqn-td-model} gives us
\begin{equation}
    0 = \frac{1}{\rho} G
    + \left(\nu+\frac{1}{2}\nu_T\right)\frac{d^2U}{dz^2}
    + \frac{1}{2}kq \frac{dU}{dz}
    \label{law-of-the-wall-eqn}
\end{equation}
where $G = -\frac{\partial\overline{p}}{\partial x}$ is the horizontal pressure
gradient, which is some constant. 

In the case $G=0$, we can solve (\ref{law-of-the-wall-eqn}) exactly to give us
\begin{equation}
    U(z) = \frac{A}{kq} \log(2\nu + kqz)
    \label{law-of-the-wall-2}
\end{equation}
where $A$ is some constant. The condition that shear stress is (approximately)
constant allows us to write $A$ in terms of the \textit{bed shear stress}, the
shear stress on $z=0$.

For $z\ll\frac{\nu}{kq}$, (\ref{law-of-the-wall-2}) says that $U(z)\propto z$,
as we would expect for a boundary layer. In this region, molecular viscosity
$\nu$ is dominant. However, for larger $z$, $U(z)$ is logarithmic in $z$:
\begin{equation}
\boxed{
    U(z) \sim \frac{u_*}{\kappa} \log\frac{z}{z_0}
    \label{law-of-the-wall}
}
\end{equation}
Equation \ref{law-of-the-wall} is the \textit{law of the wall}. Here $u_*$ is called the \textit{slip velocity} and $z_0$ the \textit{roughness height}; $\kappa\approx0.41$ is a dimensionless constant called \textit{von Karmen's constant}. 

In practice, the law of the wall does not hold when $z$ becomes too large. Far
away from the wall, other effects such as small pressure gradients affect the
flow, and imperfections in the geometry (such as other walls) may cause the
mixing length to stop being $l\sim z$. But nonetheless the law is a good
approximation for turbulent boundary layers. 

\subsection{Entrainment: Diffusion of turbulence}

Consider a turbulent patch of length scale $l$ and turbulence intensity $u'$,
surrounded by an irrotational flow outside the patch. The vortical motion within
the patch will draw in the external fluid, straining and diffusing vorticity
into it. Thus the size of the patch will increase; its boundaries move out at a
rate proportional to $u'$. 

This diffusivity of turbulence is called the \textit{turbulent diffusivity},
written $\kappa_T$, with $\kappa_T \sim u'l$. This is not to be confused with
turbulent viscosity, though both are $\sim u'l$ with similar (but not
necessarily equal) constants of proportionality.

This diffusivity of turbulence can explain the growth of the Rayleigh-Taylor
instability in a tall tube. 

\section{Mixing}

\subsection{What is mixing?}

Mixing is the blending of fluid particles with different properties, such as
density, salinity or temperature. (These properties are related to each other:
the density of salt water increases with salinity and decreases with
temperature.) There are two processes responsible for mixing:
\begin{itemize}
    \item \textit{Stirring} is the intermingling of fluid particles of different
        properties. This produces large gradients in these properties.
    \item \textit{Diffusion} drives a flux down a gradient that reduces these
        gradients between adjacent fluid particles.
\end{itemize}
Diffusion is an irreversible process: there is a `preferred' direction for
fluxes, down a gradient; but stirring (on its own) is reversible. Diffusion is a
slow process that occurs at the molecular level, whereas stirring can happen
quickly and over large lengthscales. Together, stirring and diffusion can cause
irreversible mixing very quickly.

In general, a scalar property $S$ is governed by the advection-diffusion equation
\begin{equation}
    \DDt{S} = \kappa\grad^2S
\end{equation}
where $\kappa$ (dimensions $L^2T^{-1}$) is the molecular diffusivity of $S$.
This is often very small: the diffusivity of heat in air is
$\approx10^{-5}m^2s^{-1}$ and the diffusivity of salt in water is
$\approx10^{-9}m^2s^{-1}$. The timescale for diffusing across a length $l$ is
therefore $t\sim l^2/\kappa$, which is huge: heat takes a day to diffuse across
a metre in air, and salinity takes thirty years to diffuse a metre.

However, stirring (represented by the advection term) can help to create large
gradients in $S$, which increases the rate of diffusion (by reducing $l$).

\subsection{The energy budget}

Consider an incompressible Boussinesq fluid with a linear equation of
state.\footnote{Incompressibility means that $\divg\bs{u}=0$, but this does not
mean that the density $\rho$ is constant; $\rho$ is governed by the
advection-diffusion equation.}
Assume that dissipative heating is unimportant, and ignore any heating effects
from dilution, chemical potential energy, and so on. 

Begin with the momentum equation for a variable-density fluid:
\begin{equation}
    \DDt{(\rho\bs{u})} = -\grad p - \rho g \bs{e}_z + \rho\nu\grad^2\bs{u}
\end{equation}
and dot with $\bs{u}$. Since
\begin{equation}
    \bs{u}\cdot\grad p = \divg(p\bs{u})
\end{equation}
for an incompressible fluid, and 
\begin{equation}
    \bs{u}\cdot\rho g \bs{e}_z = \DDt{}(\rho gz) - \divg(gz\kappa\grad\rho) +
    g\kappa\frac{\partial\rho}{\partial z}
\end{equation}
(where we have used the advection-diffusion equation for $\rho$, with mass
diffusivity $\kappa$), we have:
\begin{equation}
    \DDt{}\left(\frac{1}{2}\rho|\bs{u}|^2 + \rho gz\right)
    + \divg\left(p\bs{u} - \frac{1}{2}\rho\nu\grad|\bs{u}|^2 
       - gz\kappa\grad\rho\right)
    = -g\kappa\frac{\partial\rho}{\partial z} 
    - \rho\nu|\grad\bs{u}|^2.
    \label{mixing-energy-eqn}
\end{equation}
This equation describes what happens to the kinetic and potential energy density of a fluid. The energy of a parcel of fluid changes due to diffusion, because:
\begin{itemize}
    \item work is done on it by pressure from neighbouring parcels, causing an
        energy flux $p\bs{u}$; 
    \item viscosity causes the diffusion of kinetic energy, with energy flux
        $-\frac{1}{2}\rho\nu\grad|\bs{u}|^2$;
    \item mass diffusion causes the diffusion of potential energy, with energy
        flux $- gz\kappa\grad\rho$.
\end{itemize}
The energy of a parcel also changes due to:
\begin{itemize}
    \item mass diffusion raising the centre of mass, changing the potential
        energy at a rate $-g\kappa\frac{\partial\rho}{\partial z}$;
    \item dissipation due to viscosity, decreasing the kinetic energy at a rate
        $\rho\nu|\grad\bs{u}|^2$.
\end{itemize}
These processes, whose terms appear on the RHS of (\ref{mixing-energy-eqn}), are
irreversible.

If there are no fluxes of mass or momentum from the boundaries of $V$, then
integrating the energy equation gives us:
\begin{equation}
    \frac{d}{dt}(KE + PE) = W-\epsilon
\end{equation}
where $KE = \int_V \frac{1}{2}\rho|\bs{u}|^2 dV$ and $PE = \int_V \rho gz dV$ be
the total kinetic and potential energy of the system, $W = -\int_S p
\bs{u}\cdot\bs{n} dS$ is the rate of pressure working and $\epsilon = \int_V
\rho\nu\grad|\bs{u}|^2 dV$ is the rate of dissipation.

Energy is dissipated into heat. (We are ignoring any effects due to heating.) Kinetic and potential energy can be converted into each other, but mixing and dissipation cause irreversible. 

\section{*Stably stratified flows}
Not covered.
\subsection{Stratification modifies turbulence}

\section{*Mixing efficiency}
Not covered.

\section{*Internal mixing}
Not covered. 
\subsection{Kelvin-Helmholtz instability}

\subsection{Stratified shear flow}
\subsection{Holmboe instability}
\subsection{Entrainment by internal mixing}
