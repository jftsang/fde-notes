\section{Particle-laden and granular flow}

We consider the flow of a current consisting of a fluid with a well-mixed
suspension of particles, with the particles denser than the fluid. The
difference in density means that \textit{buoyancy} has a role to play. The flow is
driven by a pressure gradient and affected by the effects of buoyancy. Viscosity
is negligible, so buoyancy and inertia balance each other. 

Particles can settle out of suspension (\textit{sedimentation}) or be \textit{entrained}
into suspension. This means the concentration of the particles in the suspension
changes. So the density difference between the suspension and the ambient fluid
changes too. 

\subsection{Types of particulate flows}

We model four types of particulate gravity currents. The models are ad-hoc and
apply to different geophysical situations, where different assumptions may be
made about the flows and the particles:

\paragraph{Grain suspension by fluid turbulence} Examples include pyroclastic
flows
%\footnote{Wikipedia: A pyroclastic flow is a fast-moving current of hot gas and
%rock (collectively known as tephra), which reaches speeds moving away from a
%volcano of up to 700 km/h (450 mph). The gas can reach temperatures of about
%1,000\deg C. Pyroclastic flows normally hug the ground and travel downhill, or
%spread laterally under gravity. Their speed depends upon the density of the
%current, the volcanic output rate, and the gradient of the slope. They are a
%common and devastating result of certain explosive volcanic eruptions.}
, turbidity currents
\footnote{Wikipedia: A turbidity current is a current of rapidly moving,
sediment-laden water moving down a slope through water, or another fluid. The
current moves because it has a higher density than the fluid through which it
flows—the driving force of a turbidity current derives from its sediment, which
renders the turbid water denser than the clear water above. The deposit of a
turbidity current is called a turbidite.}
and powder snow
\footnote{Wikipedia: Freshly fallen, uncompacted snow. The density and moisture
content of powder snow can vary widely; snowfall in coastal regions and areas
with higher humidity is usually heavier than a similar depth of snowfall in an
arid or continental region. Light, dry (low moisture content, typically 4–7\%
water content) powder snow is prized by skiers and snowboarders. It is often
found in the Rocky Mountains of North America and in most regions in Japan.}. 
In such flows, the particles may be very concentrated and denser than the fluid, but
they are kept in suspension by the turbulent flow of the fluid; any particles
which settle quickly undergo \textit{resuspension}.

\paragraph{Liquefied and fluidised flow} Examples include dense snow avalanches
and \textit{some} mudflows. In these cases, the flow is dominant and the
particles move with the flow; in this regime, particles do not interact with
each other (\textit{cohesionlessness}), as they are separated from each other
(\textit{grain dispersion})). Their effect is to increase the viscosity of the
fluid. However, this limits the possibility of turbulence, and so particles can 
settle out at the base of the fluid.

\paragraph{Dry grain flow and interactions} Examples include sand dune
avalanches and rock slides. Now the particles interact with each other,
colliding fequently; the fluid acts as a lubricant. 

\paragraph{Grain-supported matrix} Examples include debris flows and lahars, or
large boulders suspended in a muddy matrix. Now cohesion has a large r\^ole to
play. The flow of mud and slurries are non-Newtonian. Effects of cohesion,
friction, pore pressure, etc. may be modelled using other stress-shear rate
relationships (\textit{rheological models}), including power laws, Bingham
plastic models nad Herschel-Bulkley models. In the latter two models, the fluid
has a \textit{yield strength}: a minimum shear stress must be applied before any
flow occurs. 

\subsection{Modelling a particulate gravity current}



\subsection{Physics within a particle gravity current}

A particle gravity current can be modelled as having a main body and a base. In
the main body, there is a well-mixed suspension of particles in a turbulent
flow; in the base, the particles and fluid have low velocity and particles are
settling out. 

Let
\begin{itemize}
 \item $\rho_0$ denote the density of the fluid;
 \item $\mu_0$ denote the viscosity of the fluid;
 \item $\rho_p$ denote the density of the particles;
 \item $\phi$ denote the concentration of particles.
\end{itemize}
Then the \textit{bulk density} of the fluid-particle mixture is
\begin{equation}
    \rho_l = \rho_0 + \phi (\rho_p - \rho_0) 
\end{equation}
and the \textit{reduced gravity} is
\begin{equation}
 g' = g \frac{\rho_l - \rho_0}{\rho_0} = \phi g \frac{\rho_p - \rho_0}{\rho_0} = \phi g_0'
\end{equation}
where
\begin{equation}
    g_0' = g \frac{\rho_p - \rho_0}{\rho_0}.
\end{equation}

\subsection{Sedimenting particle-laden flows}

We now assume that particles are small compared to the scale of motion, so that the presence of an individual particle has a negligible effect on the bulk flow. However, the collective of particles will have an effect. We also assume that particles sediment slowly, over a timescale much larger than that of the flow; and that particles are numerous and well-mixed, so that we can talk of the `concentration' of particles, rather than having to study individual particles.

Note that although we use the word `particles', all of this theory could just as well apply to bubbles or to droplets; a bubble can be regarded as a particle with density negligible compared to that of the fluid; the reduced gravity is negative.

The suspension changes the bulk properties of the fluid. As discussed, the suspension has density
\begin{equation}
        \rho_l = \rho_0 + \phi (\rho_p - \rho_0)
\end{equation}
where $\phi$ is the concentration by volume. We model the viscosity of the suspension as having a power law dependence on the concentration: 
\begin{equation}
 \mu = \mu_0 \left( 1 - \frac{\phi}{\phi_{max}} \right)^{-n \phi_{max}}
\end{equation}
where $\phi_{max}$ is the maximum suspension concentration that can be supported. For spherical particles, $n = \frac{5}{2}$.
\footnote{Why?}

We define the \textit{particle Reynolds number}
\begin{equation}
    \mathrm{Re}_p = \frac{\rho_l D u_s}{\mu}
\end{equation}
which governs many properties of the flow. Here $D$ is the diameter of a particle, and $u_s$ is the settling velocity, which we will determine. 

When particles sediment out of suspension, we are left with a \textit{clarified} fluid, devoid of particles, and a concentrated suspension at the bottom where all the particles gather. 

Sedimenting flows can exhibit four types of behaviour, depending on $\mathrm{Re}_p$ and on the concentration $\phi$:
\begin{itemize}
    \item Type I: Free settling of individual particles
    \item Type II: Flocculant settling: Coaleescence of particles
    \item Type III: Hindered (zone) settling: Restricted, fluid motion
    \item Type IV: Compression settling: Mechanical support
\end{itemize}
We will discuss Types I and III in detail. Types II and IV involve complicated particle-particle interactions. 

\subsubsection{Type I: Free settling of individual particles} 

This occurs when particles are very well-separated from each other, and so do not interact with each other. This occurs at very low concentrations (the \textit{dilute limit}).

\paragraph{The particle Reynolds number} When a sphere of diameter $D$ moves at a speed $u_s$ through otherwise unmoving fluid, its behaviour depends on the particle Reynolds number $\mathrm{Re}_p=\frac{\rho_lDu_s}{\mu}$.  If $\mathrm{Re}_p\ll1$ then the flow past the sphere is \textit{laminar}; the sphere travels smoothly. But if $10^3 \ll \mathrm{Re}_p \ll 2\cdot10^5$ then the flow is in the \textit{inertial regime}; the sphere travels roughly, and the boundary layer on the sphere is turbulent. Various other regimes are possible for different values of $\mathrm{Re}_p$; they are discussed in Middleton and Southard (1984). 

\paragraph{The drag coefficient} By dimensional analysis, the drag $F_D$ on a particle travelling at $u_s$ through a fluid is
\begin{equation}
 F_D = C_D\frac{1}{2}\rho_lu_s^2A_p
\end{equation}
where $A_p$ is the planar area of the particle. (See Prandtl and Tietjens, 1957 for details.) The dimensionless coefficient $C_D$ is called the \textit{drag coefficient}, and depends on the Reynolds number. The dependence may be found empirically:
\begin{itemize}
	\item When $\mathrm{Re}_p\ll1$, then $C_D$ is inversely proportional to $\mathrm{Re}_p$:
	\begin{equation}
    		C_D = \frac{24}{\mathrm{Re}_p}
	\end{equation}
	\item When $0.2 < \mathrm{Re}_p < 10^3$, there is a transitional region.
	\item When $10^3 < \mathrm{Re}_p < 2\cdot10^5$, $C_D$ is approximately constant for cylinders, spheres and discs. For these shapes, $C_D \approx 0.44$.
	\item At $\mathrm{Re}_p\approx2\cdot10^5$, the drag coefficient drops very quickly before increasing much more slowly again. This is called the \textit{drag crisis regime}, and arises because of a flow separation.
\end{itemize}

\paragraph{The settling velocity} So what is the settling velocity $u_s$? We consider the momentum balance on a particle in suspension. The forces acting on the particle are drag, buoyancy and gravity, and so:
\begin{align}
    m\dod{u}{t} &= F_g - F_b - F_D \\
    F_D            &= C_D\frac{1}{2}\rho_lu^2A_p \\
    F_b            &= \rho_lgV \\
    F_g            &= \rho_pgV
\end{align}
where $V$ is the volume of the particle. The expression for $F_b$ is given by Archimedes' law: the buoyancy force is $g$ times the mass of fluid displaced. (Note that the fluid displaced is said to have density $\rho_l$ rather than $\rho_0$, though the two are approximately equal in the dilute limit.) Setting $\od{u}{t} = 0$ and solving for $u$ gives us the terminal velocity, which is the settling velocity.

For a sphere of diameter $D$, we have $V = \frac{1}{6}\pi D^3$ and $A_p = \frac{1}{4}\pi D^2$, and so
\begin{equation}
    u_s=\sqrt{\frac{4g(\rho_p-\rho_l)D}{3C_D\rho_l}}.
\end{equation}
Recall that in the Stokes regime with $\mathrm{Re}_p < 0.2$, we have $C_D=\frac{24}{\mathrm{Re}_p}$. Hence:
\begin{align}
    F_D &= 3\pi u_s \mu D \\
    u_s &= \frac{g(\rho_p-\rho_l)D^2}{18\mu}
\end{align}
with $u_s \propto D^2$ and dependent on $\mu$. But in the inertial, turbulent regime, with $10^3 < \mathrm{Re}_p < 2\cdot10^5$, we have $C_D=0.44$ constant, and 
\begin{align}
    F_D &= 0.055\pi\rho_lu_s^2D^2 \\
    u_s &= 1.74\sqrt{\frac{g(\rho_p-\rho_l)D}{\rho_l}}
\end{align}
with $u_s \propto D^{1/2}$ and not dependent on $\mu$.

These predictions are confirmed by observations: the Stokes regime holds for fine grains such as silt, whereas the inertial regime holds for very coarse sand, granules and pebbles (provided that particle-particle interactions may be neglected). However, for $D$ between 0.1mm and 1mm (such as for medium or coarse sand), we are in the transitional regime where neither the Stokes nor the turbulent predictions hold well. Many equations have been proposed to describe the intermediate region, including the \textit{Ferguson-Church equation}.\footnote{See \url{http://hinderedsettling.com/2013/08/09/grain-settling-python/}.}

\paragraph{Advection and diffusion of particles} The concentration of particles $\phi$ is governed by an advection-diffusion equation:
\begin{equation}
	\dpd{\phi}{t} + \bs{u}\cdot\grad\phi = D\grad^2\phi
	\label{suspension-advdiff}
\end{equation}
where $D$ now denotes the diffusion coefficient, \textit{not} the particle diameter. The settling velocity which we have so far calculated was for an otherwise stationary fluid, so in (\ref{suspension-advdiff}) the particle velocity $\bs{u}$ is equal to
\begin{equation}
 	\mathbf{u} = (u, v, w) + (0, 0, -u_s)
\end{equation}
where $(u,v,w)$ is the mean velocity of the fluid.\footnote{We can have fluid flow even if there is no mean fluid flow, if we are in a turbulent regime.}

If there is no mean background fluid flow, then (\ref{suspension-advdiff}) reduces to
\begin{equation}
    \dpd{\phi}{t} - u_s\dpd{\phi}{z} = D\grad^2\phi.
    \label{suspension-advdiff-settling}
\end{equation}
If we neglect dependencies on $x$ and $y$, then (\ref{suspension-advdiff-settling}) is Burgers' equation. 

The diffusion coefficient $D$ depends on $\mathrm{Re}_p$ and whether the flow is laminar or turbulent. In the laminar regime, diffusion of particles is due to molecular (Brownian) diffusion: 
\begin{equation}
    D = D_B \sim \frac{kT}{6\pi\mu r}
\end{equation}
where $r$ is the radius of particles, $k$ is the Boltzmann constant and $T$ is the temperature. Molecular diffusion is a very slow process unless $r$ is very small, such as for aerosol particles.

However, in the turbulent regime, diffusion of particles is due to turbulent diffusivity:
\begin{equation}
    D = D_T \sim u^\star h
\end{equation}
where $u^\star = \sqrt{\frac{\tau_b}{\rho}}$ is the shear velocity (friction velocity), $\tau_b$ is the shear stress at the bottom, and $h$ is some turbulence length scale, such as the distance from the bottom.

Note that if we ignore diffusivity in Equation \ref{suspension-advdiff-settling} and set $D=0$, then the resulting equation admits shocks. This is because when the diffusivity is small, we have fronts where the concentration changes very sharply. To calculate the speed at which these fronts move, we can use the Rankine-Hugoniot conditions.

\subsubsection{Type II: Flocculent settling, particle coalescence}

In this regime, particles may coalesce together to form larger particles, which have larger settling velocities. Therefore settling in this regime occurs quicker than if particles do not coalesce. There is no mathematical theory to model this regime.

\subsubsection{Type III: Hindered (zone) settling}

When there are many particles, then particles no longer settle at the $u_s$ which we have calculated, but at some slower velocity $u_h$, a \textit{hindered settling velocity}. The hindered settling velocity depends on $\phi$ and on $u_s$. As $\phi$ increases, $u_h$ decreases. As $\phi$ approaches $\phi_{max}$, then more complicated inter-particle interactions take place and we start to have compression settling. 

The model that we will use is 
\begin{equation}
	u_h = u_s \left( 1 - \frac{\phi}{\phi_max} \right)^\alpha
\end{equation}
where $\alpha$ is some empirically determined constant which depends on the flow regime.\footnote{Following 'A physical introduction to suspension dynamics'.} Richardson and Zaki (1954) found that for $\phi$ up to 0.35, $\alpha = 4.65$ in the laminar regime $\mathrm{Re}_p < 0.2$ and $\alpha = 2.39$ in the regime $\mathrm{Re}_p > 500$. 
%TODO proper citations

The settling flux is
\begin{equation}
	Q_h = u_h \phi.
\end{equation}
In a suspension with a steady concentration profile $\phi(z)$, the settling flux of particles downwards is balanced by diffusive flux upwards. Thus the concentration satisfies the \textit{Rouse equation}:
\begin{equation}
	\phi(z) u_h(z) = -D \dod{\phi(z)}{z}
\end{equation}
where $D$ is a diffusivity. This tends to happen if the flow is turbulent and $D$ is large.

\paragraph{Nonlinear kinematic wave equation}
If $\phi = \phi(z,t)$ and diffusivity is small (which happens if the flow is laminar) then, as we have seen before, $\phi$ obeys the `traffic equation'
\begin{equation}
	\dpd{\phi}{t} + \dpd{\phi u_h}{z} = 0
\end{equation}
which may be solved by the method of characteristics. Again, solutions to this equation have shocks, which represent sharp fronts in a settling suspension.

\paragraph{Settling of particles on inclines}

\subsubsection{Type IV: Compressional settling, compacting}

This regime occurs in regions of very high concentration. Stirring or tapping can allow the liquid to escape. The packing fraction may increase locally, leading to increased stability.  
%TODO

\subsection{Sediment transport}

Particles in a fluid can move by several means. Particles which are in suspension (the \textit{suspended load}) or which have dissolved (the \textit{dissolved load}) will move with the flow. Particles which have settled to the bottom of the fluid can roll or slide along the bottom, or by hopping (\textit{saltation}). In a river, silt and clay tend to move by suspension; sand particles, which are larger, by saltation; gravel, larger still, by rolling and sliding. 

A particle out of suspension does not necessarily move when there is a flow. A rock on the bed starts to move only when the flow exerts a critical shear stress on the bed. However, if the flow is turbulent then a sudden eddy may be able to move the rock: turbulent flows are stochastic, and need to be modelled as such. 

Transport will change the morphology of the bed by erosion deposition: a strong flow can break up large rocks into smaller pieces. This in turn changes whether these particles are able to be suspended or settle out of suspension.  

\subsection{Aqueous and aeolian bedforms (dunes)}

\subsection{Dry granular flows, rheology, segregation}

Most granular materials do not fill the space that they occupy, but instead have a \textit{packing fraction} $\phi$. The packing fraction is variable for a given material: a collection of uniform rigid spheres has a maximum packing fraction of $\phi\approx0.74$, but randomly packed spheres have typically $\phi\approx0.64$. The value of the packing fraction depends on the history of the material: that is, how it came to be arranged as it is. A pile of sand could have very different packing fractions depending on whether the sand was simply poured onto a surface, or slowly added and packed together. 

A granular flow with particles of different sizes or densities will exhibit \textit{segregation}. Larger particles will tend to come to the top of the flow. We can demonstrate this by shaking a jar containing sand and marbles.




