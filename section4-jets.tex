\section{Jets}

We define 
\begin{align}
 \pi Q	&= \frac{1}{T}\int_0^T \int_0^\infty \int_{-\pi}^\pi w r \dif \theta\dif r \dif t \\ 
 \pi M	&= \frac{1}{T}\int_0^T \int_0^\infty \int_{-\pi}^\pi w^2 r \dif \theta\dif r \dif t \\  
\end{align}
as, respectively, the volume flux and momentum flux divided by density. The
integral over $t$ enacts a time-averaging. The $\pi$'s on the LHS are for
convenience. For a top hat profile, these become
\begin{align}
 \pi Q	&= \pi b^2 W \\
 \pi M	&= \pi b^2 W^2 
\end{align}
where we write $W$ instead of $w$ to emphasise that this is an averaged
velocity, reserving $w$ for the fluctuating local velocity. We have
\begin{align}
 W	&= M/Q \\
 b	&= Q/M^{1/2}
\end{align}

Consider a steady jet travelling in the $z$ direction. As the fluid in the jet
travels, it entrains the surrounding fluid with entrainment velocity $u_e$.
Experiments find that
\begin{equation}
 u_e = \alpha W 
\end{equation} 
where the entrainment coefficient $\alpha$ is between 0.065 and 0.080 for a top
hat profile. Thus, the volume flux satisfies
\begin{align}
 \pi \dod{Q}{z}	&= 2\pi b u_e \\
 			&= 2\pi b \alpha W \\
			&= 2\alpha\pi M^{1/2}.
\end{align}
Meanwhile, conservation of momentum gives
\begin{equation}
 \pi \dod{M}{z} = 0.
\end{equation}
Hence $M = M_0$ is constant along the plume. This gives
\begin{align}
 Q 	&= 2\alpha M_0^{1/2} (z+z_\nu) \\
 b	&= \alpha (z+z_\nu) \\
 W	&= \frac{M_0^{1/2}} {2\alpha (z+z_\nu)}.
\end{align}
So, the jet's thickness increases linearly with $z$, and the fluid velocity decreases with $z$. The point $z = -z_\nu$ is the `virtual origin' of the jet: it looks as though there is a point source of momentum at this point, even though boundary conditions are being specified at $z=0$. 
